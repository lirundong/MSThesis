\chapter{量化网络的反向蒸馏训练} \label{chap::gq-nets}

% ==============================================================================
%  Introduction
% ==============================================================================
\section{引言}
第~\ref{chap::fqn} 章针对 FQN 模型的分析显示,基于量化神经网络的目标检测模型训练较一般模型困难,其原因在于量化误差为目标检测模型训练引入了更多不稳定性。若针对此类误差建模,并在训练过程中显式地优化该误差,则有可能得到更为鲁棒的量化目标检测模型。

本章介绍一种将神经网络量化误差以可微分形式隐式建模,并利用类似模型蒸馏训练的反向蒸馏技术,将该误差模型引入原网络训练过程的训练方法。该方法可平行整合至原模型训练方法中,将模型训练过程引导至同时最小化原损失函数和量化误差的求解点,从而得到适合被量化运行的神经网络模型。我们称使用该方法训练的量化神经网络为 GQ-Nets (Guided Quantization Networks)。

GQ-Nets 使用在相同输入下,量化神经网络和全精度浮点神经网络之间最终输出的不一致性作为量化误差的度量。在分类任务中,该不一致性由各神经网络最后一层输出(即 logits)间的 KL--散度刻画;在检测任务中,则由模型分类子网络输出的 logits 间的 KL--散度,以及回归子网络输出间的 smooth L1-loss 一同刻画。这类一致性标准虽然没有针对模型参数及量化参数的显式表达式,但通过在量化操作的反向传播过程中引入 STE,其计算过程变得可微分,从而使得量化误差能够在训练过程中被显式地优化。

在将针对量化误差的优化引入原模型训练过程后,原模型优化问题演化为一阶多目标优化问题,即需要考虑最小化量化误差的梯度是否会与原模型训练梯度冲突。通过在模型训练过程中对这两组梯度间余弦相似度的度量,揭示了这两组梯度在训练的多数过程中近似正交;即对于大部分任务而言,减少模型量化误差这一任务并不与模型原始任务冲突。基于上述观察,本章引入了调和这两组梯度的一系列训练方法,并对其效果进行了分析比较。
% ------------------------------------------------------------------------------
%    Motivation
% ------------------------------------------------------------------------------
\subsection{工作动机}
TODO
% ------------------------------------------------------------------------------
%    Contributions
% ------------------------------------------------------------------------------
\subsection{主要贡献}
TODO

% ==============================================================================
%  Modeling Quantization Errors
% ==============================================================================
\section{量化误差的建模}
TODO

% ==============================================================================
%  Method
% ==============================================================================
\section{反向蒸馏训练}
TODO
% ------------------------------------------------------------------------------
%    Loss function
% ------------------------------------------------------------------------------
\subsection{损失函数设计}
TODO
% ------------------------------------------------------------------------------
%    Optimization
% ------------------------------------------------------------------------------
\subsection{损失函数优化}
TODO
% ------------------------------------------------------------------------------
%    Multi-domain BN
% ------------------------------------------------------------------------------
\subsection{多域批量正则化}
TODO

% ==============================================================================
%  Experiments
% ==============================================================================
\section{主要实验结果}
TODO

% ==============================================================================
%  Ablation study
% ==============================================================================
\section{对比实验及分析}
TODO

% ==============================================================================
%  Conclusion
% ==============================================================================
\section{结论}
TODO

